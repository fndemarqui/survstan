\nonstopmode{}
\documentclass[letterpaper]{book}
\usepackage[times,inconsolata,hyper]{Rd}
\usepackage{makeidx}
\usepackage[utf8]{inputenc} % @SET ENCODING@
% \usepackage{graphicx} % @USE GRAPHICX@
\makeindex{}
\begin{document}
\chapter*{}
\begin{center}
{\textbf{\huge Package `survstan'}}
\par\bigskip{\large \today}
\end{center}
\inputencoding{utf8}
\ifthenelse{\boolean{Rd@use@hyper}}{\hypersetup{pdftitle = {survstan: Fitting Survival Regression Models via Stan}}}{}
\ifthenelse{\boolean{Rd@use@hyper}}{\hypersetup{pdfauthor = {Fabio Demarqui}}}{}
\begin{description}
\raggedright{}
\item[Title]\AsIs{Fitting Survival Regression Models via Stan}
\item[Version]\AsIs{0.0.1}
\item[Description]\AsIs{The R package survstan enables one to fit parametric survival regression models under the maximum likelihood approach via Stan. Implemented regression models include accelerated failure time models, proportional hazards models, proportional odds models, and accelerated hazard models. Available baseline survival distributions include exponential, Weibull, log-normal, and log-logistic distributions.}
\item[License]\AsIs{MIT + file LICENSE}
\item[Encoding]\AsIs{UTF-8}
\item[Roxygen]\AsIs{list(markdown = TRUE)}
\item[RoxygenNote]\AsIs{7.2.3}
\item[Biarch]\AsIs{true}
\item[Depends]\AsIs{R (>= 3.4.0),
survival}
\item[Imports]\AsIs{actuar (>= 3.0.0),
dplyr,
ggplot2,
gridExtra,
methods,
Rcpp (>= 0.12.0),
RcppParallel (>= 5.0.1),
Rdpack,
rlang,
rstan (>= 2.18.1),
rstantools (>= 2.3.1),
tibble}
\item[RdMacros]\AsIs{Rdpack}
\item[LinkingTo]\AsIs{BH (>= 1.66.0),
Rcpp (>= 0.12.0),
RcppEigen (>= 0.3.3.3.0),
RcppParallel (>= 5.0.1),
rstan (>= 2.18.1),
StanHeaders (>= 2.18.0)}
\item[SystemRequirements]\AsIs{GNU make}
\item[URL]\AsIs{}\url{https://github.com/fndemarqui/survstan}\AsIs{,
}\url{https://fndemarqui.github.io/survstan/}\AsIs{}
\item[BugReports]\AsIs{}\url{https://github.com/fndemarqui/survstan/issues}\AsIs{}
\item[Suggests]\AsIs{knitr,
rmarkdown,
testthat (>= 3.0.0)}
\item[Config/testthat/edition]\AsIs{3}
\item[VignetteBuilder]\AsIs{knitr}
\end{description}
\Rdcontents{\R{} topics documented:}
\inputencoding{utf8}
\HeaderA{survstan-package}{The 'survstan' package.}{survstan.Rdash.package}
\aliasA{survstan}{survstan-package}{survstan}
%
\begin{Description}\relax
The aim of the R package survstan is to provide a toolkit for fitting survival models using Stan. The R package survstan can be used to fit right-censored survival data under independent censoring. The implemented models allow the fitting of survival data in the presence/absence of covariates. All inferential procedures are currently based on the maximum likelihood (ML) approach.
\end{Description}
%
\begin{References}\relax
Stan Development Team (2023).
``RStan: the R interface to Stan.''
R package version 2.21.8, \url{https://mc-stan.org/}.

Lawless JF (2002).
\emph{Statistical Models and Methods for Lifetime Data},  Wiley Series in Probability and Statistics, 2nd Edition edition.
John Wiley and Sons.
ISBN 9780471372158.

Bennett S (1983).
``Analysis of survival data by the proportional odds model.''
\emph{Statistics in Medicine}, \bold{2}(2), 273-277.

Chen YQ, Wang M (2000).
``Analysis of Accelerated Hazards Models.''
\emph{Journal of the American Statistical Association}, \bold{95}(450), 608-618.
\Rhref{https://doi.org/10.1080/01621459.2000.10474236}{doi:10.1080\slash{}01621459.2000.10474236}, \url{https://www.tandfonline.com/doi/abs/10.1080/01621459.2000.10474236}.
\end{References}
\inputencoding{utf8}
\HeaderA{aftreg}{Fitting Accelerated Failure Time Models}{aftreg}
%
\begin{Description}\relax
Function to fit accelerated failure time (AFT)  models.
\end{Description}
%
\begin{Usage}
\begin{verbatim}
aftreg(
  formula,
  data,
  baseline = c("exponential", "weibull", "lognormal", "loglogistic"),
  ...
)
\end{verbatim}
\end{Usage}
%
\begin{Arguments}
\begin{ldescription}
\item[\code{formula}] an object of class "formula" (or one that can be coerced to that class): a symbolic description of the model to be fitted.

\item[\code{data}] data an optional data frame, list or environment (or object coercible by as.data.frame to a data frame) containing the variables in the model. If not found in data, the variables are taken from environment(formula), typically the environment from which function is called.

\item[\code{baseline}] the chosen baseline distribution; options currently available are: exponential, weibull, lognormal and loglogistic distributions.

\item[\code{...}] further arguments passed to other methods.
\end{ldescription}
\end{Arguments}
%
\begin{Value}
aftreg returns an object of class "aftreg" containing the fitted model.
\end{Value}
%
\begin{Examples}
\begin{ExampleCode}

library(survstan)
fit <- aftreg(Surv(futime, fustat) ~ ecog.ps + rx, data = ovarian, baseline = "weibull", init = 0)
summary(fit)


\end{ExampleCode}
\end{Examples}
\inputencoding{utf8}
\HeaderA{ahreg}{Fitting Accelerated Hazard Models}{ahreg}
%
\begin{Description}\relax
Function to fit accelerated hazard (AH) models.
\end{Description}
%
\begin{Usage}
\begin{verbatim}
ahreg(
  formula,
  data,
  baseline = c("exponential", "weibull", "lognormal", "loglogistic"),
  ...
)
\end{verbatim}
\end{Usage}
%
\begin{Arguments}
\begin{ldescription}
\item[\code{formula}] an object of class "formula" (or one that can be coerced to that class): a symbolic description of the model to be fitted.

\item[\code{data}] data an optional data frame, list or environment (or object coercible by as.data.frame to a data frame) containing the variables in the model. If not found in data, the variables are taken from environment(formula), typically the environment from which function is called.

\item[\code{baseline}] the chosen baseline distribution; options currently available are: exponential, weibull, lognormal and loglogistic distributions.

\item[\code{...}] further arguments passed to other methods.
\end{ldescription}
\end{Arguments}
%
\begin{Value}
ahreg returns an object of class "ahreg" containing the fitted model.
\end{Value}
%
\begin{Examples}
\begin{ExampleCode}

library(survstan)
fit <- ahreg(Surv(futime, fustat) ~ ecog.ps + rx, data = ovarian, baseline = "weibull", init = 0)
summary(fit)



\end{ExampleCode}
\end{Examples}
\inputencoding{utf8}
\HeaderA{AIC.survstan}{Akaike information criterion}{AIC.survstan}
%
\begin{Description}\relax
Akaike information criterion
\end{Description}
%
\begin{Usage}
\begin{verbatim}
## S3 method for class 'survstan'
AIC(object, ..., k = 2)
\end{verbatim}
\end{Usage}
%
\begin{Arguments}
\begin{ldescription}
\item[\code{object}] an object of the class survstan.

\item[\code{...}] further arguments passed to or from other methods.

\item[\code{k}] numeric, the penalty per parameter to be used; the default k = 2 is the classical AIC.
\end{ldescription}
\end{Arguments}
%
\begin{Value}
the Akaike information criterion.
\end{Value}
%
\begin{Examples}
\begin{ExampleCode}

library(survstan)
fit1 <- aftreg(Surv(futime, fustat) ~ 1, data = ovarian, baseline = "weibull", init = 0)
fit2 <- aftreg(Surv(futime, fustat) ~ rx, data = ovarian, baseline = "weibull", init = 0)
fit3 <- aftreg(Surv(futime, fustat) ~ ecog.ps + rx, data = ovarian, baseline = "weibull", init = 0)
AIC(fit1, fit2, fit3)


\end{ExampleCode}
\end{Examples}
\inputencoding{utf8}
\HeaderA{anova.survstan}{anova method for survstan models}{anova.survstan}
%
\begin{Description}\relax
Compute analysis of variance (or deviance) tables for one or more fitted model objects.
\end{Description}
%
\begin{Usage}
\begin{verbatim}
## S3 method for class 'survstan'
anova(...)
\end{verbatim}
\end{Usage}
%
\begin{Arguments}
\begin{ldescription}
\item[\code{...}] further arguments passed to or from other methods.
\end{ldescription}
\end{Arguments}
%
\begin{Value}
the ANOVA table.
\end{Value}
%
\begin{Examples}
\begin{ExampleCode}

library(survstan)
fit1 <- aftreg(Surv(futime, fustat) ~ 1, data = ovarian, baseline = "weibull", init = 0)
fit2 <- aftreg(Surv(futime, fustat) ~ rx, data = ovarian, baseline = "weibull", init = 0)
fit3 <- aftreg(Surv(futime, fustat) ~ ecog.ps + rx, data = ovarian, baseline = "weibull", init = 0)
anova(fit1, fit2, fit3)


\end{ExampleCode}
\end{Examples}
\inputencoding{utf8}
\HeaderA{coef.survstan}{Estimated regression coefficients}{coef.survstan}
%
\begin{Description}\relax
Estimated regression coefficients
\end{Description}
%
\begin{Usage}
\begin{verbatim}
## S3 method for class 'survstan'
coef(object, ...)
\end{verbatim}
\end{Usage}
%
\begin{Arguments}
\begin{ldescription}
\item[\code{object}] an object of the class survstan

\item[\code{...}] further arguments passed to or from other methods
\end{ldescription}
\end{Arguments}
%
\begin{Value}
the estimated regression coefficients
\end{Value}
%
\begin{Examples}
\begin{ExampleCode}

library(survstan)
fit <- aftreg(Surv(futime, fustat) ~ ecog.ps + rx, data = ovarian, baseline = "weibull", init = 0)
coef(fit)


\end{ExampleCode}
\end{Examples}
\inputencoding{utf8}
\HeaderA{confint.survstan}{Confidence intervals for the regression coefficients}{confint.survstan}
%
\begin{Description}\relax
Confidence intervals for the regression coefficients
\end{Description}
%
\begin{Usage}
\begin{verbatim}
## S3 method for class 'survstan'
confint(object, parm = NULL, level = 0.95, ...)
\end{verbatim}
\end{Usage}
%
\begin{Arguments}
\begin{ldescription}
\item[\code{object}] an object of the class survstan.

\item[\code{parm}] a specification of which parameters are to be given confidence intervals, either a vector of numbers or a vector of names. If missing, all parameters are considered.

\item[\code{level}] the confidence level required.

\item[\code{...}] further arguments passed to or from other methods.
\end{ldescription}
\end{Arguments}
%
\begin{Value}
100(1-alpha) confidence intervals for the regression coefficients.
\end{Value}
%
\begin{Examples}
\begin{ExampleCode}

library(survstan)
fit <- aftreg(Surv(futime, fustat) ~ ecog.ps + rx, data = ovarian, baseline = "weibull", init = 0)
confint(fit)


\end{ExampleCode}
\end{Examples}
\inputencoding{utf8}
\HeaderA{estimates}{Parameters estimates of a survstan model}{estimates}
%
\begin{Description}\relax
Parameters estimates of a survstan model
\end{Description}
%
\begin{Usage}
\begin{verbatim}
estimates(object, ...)
\end{verbatim}
\end{Usage}
%
\begin{Arguments}
\begin{ldescription}
\item[\code{object}] an object of the class survstan.

\item[\code{...}] further arguments passed to or from other methods.
\end{ldescription}
\end{Arguments}
%
\begin{Value}
the parameters estimates of a given survstan model.
\end{Value}
%
\begin{Examples}
\begin{ExampleCode}

library(survstan)
fit <- aftreg(Surv(futime, fustat) ~ ecog.ps + rx, data = ovarian, baseline = "weibull", init = 0)
estimates(fit)


\end{ExampleCode}
\end{Examples}
\inputencoding{utf8}
\HeaderA{ggresiduals}{Generic S3 method ggresiduals}{ggresiduals}
%
\begin{Description}\relax
Generic S3 method ggresiduals
\end{Description}
%
\begin{Usage}
\begin{verbatim}
ggresiduals(object, ...)
\end{verbatim}
\end{Usage}
%
\begin{Arguments}
\begin{ldescription}
\item[\code{object}] a fitted model object.

\item[\code{...}] further arguments passed to or from other methods.
\end{ldescription}
\end{Arguments}
%
\begin{Details}\relax
Generic method to plot residuals of survival models.
\end{Details}
%
\begin{Value}
the desired residual plot.
\end{Value}
\inputencoding{utf8}
\HeaderA{ggresiduals.survstan}{ggresiduals method for survstan models}{ggresiduals.survstan}
%
\begin{Description}\relax
ggresiduals method for survstan models
\end{Description}
%
\begin{Usage}
\begin{verbatim}
## S3 method for class 'survstan'
ggresiduals(object, type = c("coxsnell", "martingale", "deviance"), ...)
\end{verbatim}
\end{Usage}
%
\begin{Arguments}
\begin{ldescription}
\item[\code{object}] a fitted model object of the class survstan.

\item[\code{type}] type of residuals used in the plot: coxsnell (default), martingale and deviance.

\item[\code{...}] further arguments passed to or from other methods.
\end{ldescription}
\end{Arguments}
%
\begin{Details}\relax
This function produces residuals plots of Cox-Snell residuals, martingale residuals and deviance residuals.
\end{Details}
%
\begin{Value}
the desired residual plot.
\end{Value}
%
\begin{Examples}
\begin{ExampleCode}

library(survstan)
ovarian$rx <- as.factor(ovarian$rx)
fit <- aftreg(Surv(futime, fustat) ~ age + rx, data = ovarian, baseline = "weibull", init = 0)
ggresiduals(fit, type = "coxsnell")
ggresiduals(fit, type = "martingale")
ggresiduals(fit, type = "deviance")


\end{ExampleCode}
\end{Examples}
\inputencoding{utf8}
\HeaderA{model.matrix.survstan}{Model.matrix method for survstan models}{model.matrix.survstan}
%
\begin{Description}\relax
Reconstruct the model matrix for a survstan model.
\end{Description}
%
\begin{Usage}
\begin{verbatim}
## S3 method for class 'survstan'
model.matrix(object, ...)
\end{verbatim}
\end{Usage}
%
\begin{Arguments}
\begin{ldescription}
\item[\code{object}] an object of the class survstan.

\item[\code{...}] further arguments passed to or from other methods.
\end{ldescription}
\end{Arguments}
%
\begin{Value}
The model matrix (or matrices) for the fit.
\end{Value}
%
\begin{Examples}
\begin{ExampleCode}

library(survstan)
fit <- aftreg(Surv(futime, fustat) ~ ecog.ps + rx, data = ovarian, baseline = "weibull", init = 0)
model.matrix(fit)


\end{ExampleCode}
\end{Examples}
\inputencoding{utf8}
\HeaderA{phreg}{Fitting Proportional Hazards Models}{phreg}
%
\begin{Description}\relax
Function to fit proportional hazards (PH) models.
\end{Description}
%
\begin{Usage}
\begin{verbatim}
phreg(
  formula,
  data,
  baseline = c("exponential", "weibull", "lognormal", "loglogistic"),
  ...
)
\end{verbatim}
\end{Usage}
%
\begin{Arguments}
\begin{ldescription}
\item[\code{formula}] an object of class "formula" (or one that can be coerced to that class): a symbolic description of the model to be fitted.

\item[\code{data}] data an optional data frame, list or environment (or object coercible by as.data.frame to a data frame) containing the variables in the model. If not found in data, the variables are taken from environment(formula), typically the environment from which function is called.

\item[\code{baseline}] the chosen baseline distribution; options currently available are: exponential, weibull, lognormal and loglogistic distributions.

\item[\code{...}] further arguments passed to other methods.
\end{ldescription}
\end{Arguments}
%
\begin{Value}
phreg returns an object of class "phreg" containing the fitted model.
\end{Value}
%
\begin{Examples}
\begin{ExampleCode}

library(survstan)
fit <- phreg(Surv(futime, fustat) ~ ecog.ps + rx, data = ovarian, baseline = "weibull", init = 0)
summary(fit)



\end{ExampleCode}
\end{Examples}
\inputencoding{utf8}
\HeaderA{poreg}{Fitting Proportional Odds Models}{poreg}
%
\begin{Description}\relax
Function to fit proportional odds (PO) models.
\end{Description}
%
\begin{Usage}
\begin{verbatim}
poreg(
  formula,
  data,
  baseline = c("exponential", "weibull", "lognormal", "loglogistic"),
  ...
)
\end{verbatim}
\end{Usage}
%
\begin{Arguments}
\begin{ldescription}
\item[\code{formula}] an object of class "formula" (or one that can be coerced to that class): a symbolic description of the model to be fitted.

\item[\code{data}] data an optional data frame, list or environment (or object coercible by as.data.frame to a data frame) containing the variables in the model. If not found in data, the variables are taken from environment(formula), typically the environment from which function is called.

\item[\code{baseline}] the chosen baseline distribution; options currently available are: exponential, weibull, lognormal and loglogistic distributions.

\item[\code{...}] further arguments passed to other methods.
\end{ldescription}
\end{Arguments}
%
\begin{Value}
poreg returns an object of class "poreg" containing the fitted model.
\end{Value}
%
\begin{Examples}
\begin{ExampleCode}

library(survstan)
fit <- poreg(Surv(futime, fustat) ~ ecog.ps + rx, data = ovarian, baseline = "weibull", init = 0)
summary(fit)



\end{ExampleCode}
\end{Examples}
\inputencoding{utf8}
\HeaderA{print.summary.survstan}{Print the summary.survstan output}{print.summary.survstan}
%
\begin{Description}\relax
Print the summary.survstan output
\end{Description}
%
\begin{Usage}
\begin{verbatim}
## S3 method for class 'summary.survstan'
print(x, ...)
\end{verbatim}
\end{Usage}
%
\begin{Arguments}
\begin{ldescription}
\item[\code{x}] an object of the class summary.survstan.

\item[\code{...}] further arguments passed to or from other methods.
\end{ldescription}
\end{Arguments}
%
\begin{Value}
a summary of the fitted model.
\end{Value}
\inputencoding{utf8}
\HeaderA{summary.survstan}{Summary for a survstan object}{summary.survstan}
%
\begin{Description}\relax
Summary for a survstan object
\end{Description}
%
\begin{Usage}
\begin{verbatim}
## S3 method for class 'survstan'
summary(object, conf.level = 0.95, ...)
\end{verbatim}
\end{Usage}
%
\begin{Arguments}
\begin{ldescription}
\item[\code{object}] an object of the class 'survstan'.

\item[\code{conf.level}] the confidence level required.

\item[\code{...}] further arguments passed to or from other methods.
\end{ldescription}
\end{Arguments}
\inputencoding{utf8}
\HeaderA{tidy}{Generic S3 method tidy}{tidy}
%
\begin{Description}\relax
Generic S3 method tidy
\end{Description}
%
\begin{Usage}
\begin{verbatim}
tidy(object, conf.level = 0.95, ...)
\end{verbatim}
\end{Usage}
%
\begin{Arguments}
\begin{ldescription}
\item[\code{object}] a fitted model object.

\item[\code{conf.level}] the confidence level required.

\item[\code{...}] further arguments passed to or from other methods.
\end{ldescription}
\end{Arguments}
%
\begin{Details}\relax
Convert a fitted model into a tibble.
\end{Details}
%
\begin{Value}
a tibble with a summary of the fit.
\end{Value}
\inputencoding{utf8}
\HeaderA{tidy.survstan}{Tidy a survstan object}{tidy.survstan}
%
\begin{Description}\relax
Tidy a survstan object
\end{Description}
%
\begin{Usage}
\begin{verbatim}
## S3 method for class 'survstan'
tidy(object, conf.level = 0.95, ...)
\end{verbatim}
\end{Usage}
%
\begin{Arguments}
\begin{ldescription}
\item[\code{object}] a fitted model object.

\item[\code{conf.level}] the confidence level required.

\item[\code{...}] further arguments passed to or from other methods.
\end{ldescription}
\end{Arguments}
%
\begin{Details}\relax
Convert a fitted model into a tibble.
\end{Details}
%
\begin{Value}
a tibble with a summary of the fit.
\end{Value}
%
\begin{Examples}
\begin{ExampleCode}

library(survstan)
fit <- aftreg(Surv(futime, fustat) ~ ecog.ps + rx, data = ovarian, baseline = "weibull", init = 0)
tidy(fit)


\end{ExampleCode}
\end{Examples}
\inputencoding{utf8}
\HeaderA{vcov.survstan}{Variance-covariance matrix}{vcov.survstan}
%
\begin{Description}\relax
This function extracts and returns the variance-covariance matrix associated with the regression coefficients when the maximum likelihood estimation approach is used in the model fitting.
\end{Description}
%
\begin{Usage}
\begin{verbatim}
## S3 method for class 'survstan'
vcov(object, all = FALSE, ...)
\end{verbatim}
\end{Usage}
%
\begin{Arguments}
\begin{ldescription}
\item[\code{object}] an object of the class survstan.

\item[\code{all}] logical; if FALSE (default), only covariance matrix associated with regression coefficients is returned; if TRUE, the full covariance matrix is returned.

\item[\code{...}] further arguments passed to or from other methods.
\end{ldescription}
\end{Arguments}
%
\begin{Value}
the variance-covariance matrix associated with the parameters estimators.
\end{Value}
%
\begin{Examples}
\begin{ExampleCode}

library(survstan)
fit <- aftreg(Surv(futime, fustat) ~ ecog.ps + rx, data = ovarian, baseline = "weibull", init = 0)
vcov(fit)


\end{ExampleCode}
\end{Examples}
\printindex{}
\end{document}
